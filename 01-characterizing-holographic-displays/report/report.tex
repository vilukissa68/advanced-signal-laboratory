% !TEX encoding = UTF-8 Unicode
\documentclass[12pt,a4paper,english
% ,twoside,openright
]{tunithesis}

\special{papersize=210mm,297mm}

\author{Roosa Kuusivaara \& Väinö-Waltteri Granat}
\title{Characterizing holographic displays via
numerical simulations - Report} % primary title (for front page)
\thesistype{Laboratory Report} % or Bachelor of Science, Laboratory Report...

\usepackage{lastpage}
\usepackage[english]{babel}
\usepackage[
backend=biber,
style=authoryear,
citestyle=authoryear,
autocite=inline
]{biblatex}
\usepackage{csquotes}

\addbibresource{references.bib} %Imports bibliography file


\definecolor{tunipurple}{RGB}{78, 0, 142}

\newcommand\todo[1]{{\color{red}!!!TODO: #1}} % Remark text in braces appears in red
\newcommand{\angs}{\textsl{\AA}}              % , e.g. slanted symbol for Ångstöm
% Preparatory content ends here


\pagenumbering{roman} % was: {Roman}
\pagestyle{headings}
\begin{document}

% Special trick so that internal macros (denoted with @ in their name)
% can be used outside the cls file (e.g. \@author)
\makeatletter

% Create the title page.
% First the logo. Check its language.
\thispagestyle{empty}
\vspace*{-.5cm}\noindent

\begin{figure}
    \vspace{-1.3cm}
    \advance\leftskip-2.5cm
    \noindent\includegraphics{img/tunilogo.png}
\end{figure}
 
\vspace{2.5cm}
\begin{flushright}
\noindent\textsf{\LARGE{\@author}}

\noindent\vspace{0.5cm}

\noindent\Huge{\textsf{\textbf{\textcolor{tunipurple}{\@title}}}}
\end{flushright}
\vspace{13.7cm} % adjust to 12.7 this if thesis title needs two lines

% Last some additional info to the bottom-right corner
\begin{flushright}  
    \begin{spacing}{1.0}
      \textsf{Faculty of Information Technology and Communication Sciences (ITC)\\
      \@thesistype\\
      September 2023}
    \end{spacing}
\end{flushright}

% Leave the backside of title page empty in twoside mode
\if@twoside
\clearpage
\fi

% Turn off page numbering for the first pages
\pagenumbering{gobble}


% Some fields in abstract are automated, namely those with \@ (author,
% title, thesis type).
\chapter*{Abstract}
\begin{spacing}{1.0}
\noindent \@author: \@title\\
\@thesistype\\
Tampere University\\
Master’s Degree Programme in Signal Processing\\
September 2023
\end{spacing}
\noindent\rule{12cm}{0.4pt}

\vspace{0.5cm}

% ---------------------------------------
% Abstract and keywords
% ---------------------------------------

\noindent The abstract is a concise 1-page description of the work: what was the
problem, what was done, and what are the results. Do not include
charts or tables in the abstract.

These instructions are intended for students of Computer Sciences at the Tampere University. They cover questions of writing a thesis, such as use of the literature, structure of the thesis and style, the external appearance of the thesis and the use of tools.  These instructions do not cover the scientific content of the thesis.

~

\noindent\textbf{Keywords:} M.Sc. thesis, layout, writing style.

~


% Add the table of contents


\setcounter{tocdepth}{3}              % How many header level are included
\tableofcontents                      % Create TOC


% The actual text begins here and page numbering changes to 1,2...
% Leave the backside of title empty in twoside mode
\if@twoside
%\newpage
\cleardoublepage
\fi


\renewcommand{\chaptername}{} % This disables the prefix 'Chapter' or
                              % 'Luku' in page headers (in 'twoside'
                              % mode)


\chapter{Introduction}
\label{ch:intro}
\pagenumbering{arabic}
\setcounter{page}{1} 
In this report we describe our work with the `Characterizing holographic displays via
numerical simulations` exercise, for the Advanced Signal Processing Laboratory Course.

In this project we familiarized ourselves with the basics of holographic display, by implementing a part of a holograpics display viewing simulation. We implemented two holographic synthesis methods and a retinal image formation model, by contributing code to a MATLAB codebase given by the course faculty.

\section{Hologram synthesis}
Hologram synthesis describes the method for forming holographic images in 3d space from a given image. In this assignment, instead of using entire 3d space, the analysis can be simplified by considering only a cross section of the 3d space. The hologram is now represented as a 1D array of complex values. This simplification eases the computational load and also makes the process easier to handle.

\section{Retinal Image Formation}

\chapter{Methodology}
\section{Implementing Hologram Synthesis}
Our implementation of the holographic image viewer included three different methods of holographic image synthesis, one of which was provided in the code base given to us. The ready made synthesis method was holographic stereograph synthesis (HSS), [TODO: explain hss here]. The first synthesis method we implemented was a Fresnel hologram synthesis which is based on the Fresnel diffraction kernel. The second method we implemented was Rayleigh-Sommerfeld synthesis (RSS), which uses Rayleigh-Sommerfeld diffraction kernel. This formula is similar to the Fresnel but has some differences in calculating the hologram.
\section{Field Propagation}
\label{sec:methodology}

\chapter{Results}
\label{sec:results}
In this section we anaylze the images produced by the implemented model.

\chapter{Conclusions}
\label{ch:conclusions}


%
% The bibliography, i.e the list of references
%
\newpage

\printbibliography[title=References]
\addcontentsline{toc}{chapter}{References}

\end{document}

