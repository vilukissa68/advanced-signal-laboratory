% !TEX encoding = UTF-8 Unicode
\documentclass[12pt,a4paper,english
% ,twoside,openright
]{tunithesis}

\special{papersize=210mm,297mm}

\author{Roosa Kuusivaara \& Väinö-Waltteri Granat}
\title{Image Recognition - Report} % primary title (for front page)
\thesistype{Laboratory Report} % or Bachelor of Science, Laboratory Report...

\usepackage{lastpage}
\usepackage[english]{babel}
\usepackage[
backend=biber,
style=authoryear,
citestyle=authoryear,
autocite=inline
]{biblatex}
\usepackage{csquotes}

\addbibresource{references.bib} %Imports bibliography file


\definecolor{tunipurple}{RGB}{78, 0, 142}

\newcommand\todo[1]{{\color{red}!!!TODO: #1}} % Remark text in braces appears in red
\newcommand{\angs}{\textsl{\AA}}              % , e.g. slanted symbol for Ångstöm
% Preparatory content ends here


\pagenumbering{roman} % was: {Roman}
\pagestyle{headings}
\begin{document}

% Special trick so that internal macros (denoted with @ in their name)
% can be used outside the cls file (e.g. \@author)
\makeatletter

% Create the title page.
% First the logo. Check its language.
\thispagestyle{empty}
\vspace*{-.5cm}\noindent

\begin{figure}
    \vspace{-1.3cm}
    \advance\leftskip-2.5cm
    \noindent\includegraphics{img/tunilogo.png}
\end{figure}
 
\vspace{2.5cm}
\begin{flushright}
\noindent\textsf{\LARGE{\@author}}

\noindent\vspace{0.5cm}

\noindent\Huge{\textsf{\textbf{\textcolor{tunipurple}{\@title}}}}
\end{flushright}
\vspace{13.7cm} % adjust to 12.7 this if thesis title needs two lines

% Last some additional info to the bottom-right corner
\begin{flushright}  
    \begin{spacing}{1.0}
      \textsf{Faculty of Information Technology and Communication Sciences (ITC)\\
      \@thesistype\\}
    \end{spacing}
\end{flushright}

% Leave the backside of title page empty in twoside mode
\if@twoside
\clearpage
\fi

% Turn off page numbering for the first pages
\pagenumbering{gobble}


% Some fields in abstract are automated, namely those with \@ (author,
% title, thesis type).
\chapter*{Abstract}
\begin{spacing}{1.0}
\noindent \@author: \@title\\
\@thesistype\\
Tampere University\\
Master’s Degree Programme in Signal Processing\\
October 2023 \\
\end{spacing}
\noindent\rule{12cm}{0.4pt}

\vspace{0.5cm}

% ---------------------------------------
% Abstract and keywords
% ---------------------------------------

\noindent
This report documents the work done in the Image Recognition assignment as a part of the Advanced Signal Processing Laboratory course. In the assignment we familiarize ourselves with modern machine learning, in particular deep learning, and apply them to the task of building a smile detector for real-time execution. The goal is to achieve an accuracy of at least $85 \%$ in classifying images based on facial expressions, smiles or non-smiles, using GENKI-4k dataset for training the network.


~

\noindent\textbf{Keywords:} Laboratory Report, Machine Learning, Deep Learning, Image Recognition


% Add the table of contents


\setcounter{tocdepth}{3}              % How many header level are included
\tableofcontents                      % Create TOC


% The actual text begins here and page numbering changes to 1,2...
% Leave the backside of title empty in twoside mode
\if@twoside
%\newpage
\cleardoublepage
\fi


\renewcommand{\chaptername}{} % This disables the prefix 'Chapter' or
                              % 'Luku' in page headers (in 'twoside'
                              % mode)


\chapter{Introduction}
\label{ch:intro} 
This is an introduction
\pagenumbering{arabic}
\setcounter{page}{1} 

\chapter{Methodology}
\label{sec:methodology}

\chapter{Results}
\label{sec:results}

\chapter{Conclusions}
\label{ch:conclusions}

%
% The bibliography, i.e the list of references
%
\newpage

\printbibliography[title=References]
\addcontentsline{toc}{chapter}{References}

\end{document}

